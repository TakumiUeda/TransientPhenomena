% !TEX root =../electromagnetics.tex
\usepackage[framed,thmmarks]{ntheorem}

\usepackage{framed}
\usepackage{color}


\theorembodyfont{\normalfont}
\theoremstyle{break}
\theoremseparator{.}
\newtheorem{Definition}{定義}[section]
\theoremclass{Definition}
\theoremstyle{break}
\newframedtheorem{mydef}[Definition]{定義}

\theorembodyfont{\normalfont}
\theoremstyle{break}
\theoremseparator{.}
\newtheorem{Theorem}{定理}[section]
\theoremclass{Theorem}
\theoremstyle{break}
\newframedtheorem{mytheorem}[Theorem]{定理}


\newtheorem{Axion}{公理}[section]
\newtheorem{plac}{演習}[section]
\newtheorem{ans}{解}[section]
\newtheorem{Proposition}{命題}[section]
\newtheorem{exam}{例}[section]
\newtheorem{Proof}{導出や解説}[section]




%%% real field R
\newcommand{\Rfield}{\mbox{\bf R}}
%%% integer field Z
\newcommand{\Ifield}{\mbox{\bf Z}}
%%% positive integer field Z+
\newcommand{\PIfield}{\mbox{\bf Z$^{+}$}}
%%% natual integer field N
\newcommand{\NIfield}{\mbox{\bf N}}
%%%
\newcommand{\Qfield}{\mbox{\bf Q}}

%%%  grad, div, rot 記号
\newcommand{\vgrad}[1]{\mathrm{grad } #1}
\newcommand{\vdiv}[1]{\mathrm{div } \vecbm{#1}} 
\newcommand{\vrot}[1]{\mathrm{rot } \vecbm{#1}}
\newcommand{\lhaplus}[1]{\bm{\nabla}^{2} \bm{#1}}
\newcommand{\cross}[2]{\bm{#1} \times \bm{#2}}
\newcommand{\scross}[2]{\bm{a} \times \bm{a}}
\newcommand{\idot}[2]{\bm{#1} \cdot \bm{#2}}


%%%  1/2
\newcommand{\half}{\frac{1}{2}}

   %%%  逆ベクトルアクセント
   \newcommand{\ivec}[1]{\stackrel{\leftarrow}{#1}}

   %%%  定義記号
   \newcommand{\defeq}{\stackrel{\mbox{\footnotesize def}}{=}}

   %%%  逆数
   \newcommand{\rev}[1]{\dfrac{1}{#1}}

   %%%  偏微分
   \newcommand{\pdrv}[2]{\dfrac{\partial #1}{\partial #2}}
   \newcommand{\rdrv}[2]{\dfrac{D #1}{D #2}}
   \newcommand{\spdrv}[2]{\dfrac{\partial^{2} #1}{\partial^{2} #2}}

   %%%  微分
   \newcommand{\drv}[2]{\dfrac{d#1}{d#2}}
   %%%  2階微分
   \newcommand{\ddrv}[2]{\dfrac{d^{2}#1}{d{#2}^{2}}}

   %%%  変分
   %\newcommand{\drvd}[2]{\dfrac{\delta #1}{\delta #2}}

   %%%  < (ブラ)と > (ケット)
   \newcommand{\lag}{\langle}
   \newcommand{\rag}{\rangle}


   %%%  boldmath で出力
   \newcommand{\vecbm}[1]{\bm{#1}}  
   \newcommand{\unit}[1]{\mathrm{[ #1 ]}}
   \newcommand{\tr}{{}^{t} }

