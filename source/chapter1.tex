% !TEX root =../main.tex
\chapter{序論}
\chapter{ラプラス変換}
\section{ラプラス変換の定義}
\begin{Definition}[ラプラス変換]
\begin{align}
\begin{split}
F(s) = \int_{0}^{\infty} f (t) e^{-st} dt
\end{split}
\end{align}
正確には
\begin{align}
\begin{split}
F(s) = \lim_{\alpha \to \infty} \int_{0}^{\alpha} f (t) e^{-st} dt
\end{split}
\end{align}
で定義される。
\end{Definition}

\section{逆ラプラス変換}

\begin{Definition}[ブロムウィッチ積分]
\begin{align}
\begin{split}
F(s) = \lim_{p \to \infty} \frac{1}{2 \pi j} \int_{c-jp}^{c+jp} F (s) e^{st} dt
\end{split}
\end{align}
\end{Definition}

\section{ラプラス変換表}
\begin{table}[htbp]
\begin{center}
\caption{ラプラス変換表}
\begin{tabular}{|c|c|c|}
\hline 
 No &  $f(t)$	& $F(s)$ \\ \hline
1& $1$		& $\dfrac{1}{s}$  \\[1.5ex]
2& $t	$	& $\dfrac{1}{s^{2}}$  \\[1.5ex]
3& $t^{n}$		& $\dfrac{n!}{s^{n+1}}$  \\[1.5ex]
4& $e^{at}$		& $\dfrac{1}{s-a}$  \\[1.5ex]
5& $\sin{at}$		& $\dfrac{a}{s^{2}+a^{2}}$  \\[1.5ex]
6& $\cos{at}$		& $\dfrac{s}{s^{2}+a^{2}}$  \\[1.5ex]
7& $e^{bt}\sin{at}$		& $\dfrac{a}{(s-b)^{2}+a^{2}}$  \\[1.5ex]
8& $e^{bt}\cos{at}$		& $\dfrac{s-b}{(s-b)^{2}+a^{2}}$  \\[1.5ex]
\hline
\end{tabular}
\end{center}
\end{table}
\begin{Proof}
\begin{align}
\begin{split}
F(s) &= \lim_{p \to \infty} \int_{0}^{p} u (t) e^{st} dt\\
&=\lim_{p \to \infty} \int_{0}^{p} e^{-st}dt\\
&=\left[ \frac{e^{-st}}{-s} \right]^{\infty}_{0}\\
&=\frac{1}{s}-\lim_{t \to \infty} \frac{e^{-st}}{s}\\
&=\frac{1}{s}
\end{split}
\end{align}
\end{Proof}

キルヒホッフの法則

\chapter{真空管}
ブラウン管
\chapter{過渡現象}
\section{過渡現象}

\begin{align}
\begin{split}
v_{L}= L \drv{i(t)}{t} \mathrm{[V]}
\end{split}
\end{align}

\begin{align}
\begin{split}
v_{C}= \frac{1}{C} \int i(t) dt \mathrm{[V]}
\end{split}
\end{align}
時定数\\


\ \\ 
RL直列回路の過渡現象
\begin{align}
\begin{split}
i=\frac{E}{R} (1-e^{-\frac{R}{L}t}) \mathrm{[A]}\\
\end{split}
\end{align}
RC直列回路の過渡現象
\begin{align}
\begin{split}
i=\frac{E}{R} e^{-\frac{1}{CR}t} \mathrm{[A]} \\
\end{split}
\end{align}

RLC直列回路の過渡現象
\begin{align}
\begin{split}
v_{t} = Ri(t) + L \drv{i(t)}{t} + \frac{1}{C} \int i dt \mathrm{[V]}
\end{split}
\end{align}
時定数\\

